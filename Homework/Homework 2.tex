\documentclass[boxes]{rutgers_hw}
\usepackage{rutgers}
% \usepackage[none]{hyphenat} % Use to avoid hyphens

\author{Rohan Karamel}
\netid{rak218} 
\assignment{Homework 2}
\date{\today}
\course{Honors Real Analysis}
\semester{Spring 2024}
\instructor{Professor Sagun Chanillo}
\institution{Rutgers University}

\newtheorem*{lemma}{Lemma}

\begin{document}


  \maketitle

  % Problem 1 Part A

  \begin{exern}{1a}
    
    Let $a_n \ge 0$. Assume that
    \[ \sum_{n=1}^\infty a_n < \infty\]
    that is the series is \textbf{convergent}. Prove the series
    \[ \sum_{n=1}^\infty a_n^p, 1 \le p < \infty\]
    is also convergent.

  \end{exern}

  \begin{proof}

    Since $\sum a_n$ is convergent, we have 
    \[ \text{(1) \hspace{1 mm}} 0 \le a_n < 1, \forall n \ge n_0\]
    and because the sum is less than infinity, for some natural number M
    \[ \text{(2) \hspace{1 mm}}\sum_{n=1}^\infty a_n < M \]
    Similarly, from (1) and because p is greater than or equal to 1,
    \[ 0 \le a_n^p < a_n \]
    Now by applying the infinite series to the inequality and utilizing (2)
    \[ 0 \le \sum_{n=1}^\infty a_n^p \le \sum_{n=1}^\infty a_n \le M\]
    And, we finally have
    \[ 0 \le \sum_{n=1}^\infty a_n^p  \le M\]
    Therefore, the series is convergent

  \end{proof}

  % Problem 1 Part B
  
  \begin{exern}{1b}
    
    Let $a_n \ge 0$. Prove if for some $p > 1$, the series $\sum_{n=1}^\infty a_n^p$ is divergent,
    then $\sum_{n=1}^\infty a_n$ is also divergent.

  \end{exern}

  \begin{proof}

    Assume for the sake of contradiction that the series $\sum_{n=1}^\infty a_n^p$ is divergent, 
    but the series $\sum_{n=1}^\infty a_n$ is convergent.

    We have already reached a contradiction, as in Problem 1a, we have shown that for all $p \ge 1$
    \[\sum_{n=1}^\infty a_n \text{ convergent} \implies \sum_{n=1}^\infty a_n^p \text{ convergent}\]
    
    Therefore, the statement must be true.

  \end{proof}

  \pagebreak

  % Problem 2

  \begin{exern}{2}

    Decide whether the following propositions are true or false, providing a short justification for each conclusion

    % Problem 2 Part A

    \begin{exern}{2A}

        If every proper subsequence of $x_n$ converges, then $x_n$ converges as well.
    
    \end{exern}

    \begin{proof}
        
        True. We know that a sequence converges if and only if all subsequences converge.
        By using the reverse direction, we have that
        \[ x_n \text{ convergent} \implies \text{all subsequences of } x_n \text{ is convergent}\]
        We can limit this to just proper sequences which is a simpler statement because we exclude the case where they are equal.
        Therefore, we have the original statement.

    \end{proof}

    % Problem 2 Part B

    \begin{exern}{2B}

        If $x_n$ contains a divergent subsequence, then $x_n$ diverges.
    
    \end{exern}

    \begin{proof}
        
        True. Assume for the sake of contradiction that $x_n$ contains a divergent subsequence, but $x_n$ itself, converges.
        Then, a contradiction forms as $x_n$ converges if and only if all subsequences of $x_n$ converge.
        Thus, the statement is true.

    \end{proof}

    % Problem 2 Part C

    \begin{exern}{2C}

        If $x_n$ is bounded and diverges, then there exist two subsequences of $x_n$ that converge to different limits.
    
    \end{exern}

    \begin{proof}
        
        True. Since the sequence is divergent, we know that $\limsup_{n\to\infty} x_n \neq \liminf_{n\to\infty} x_n$.
        However, by definition, there exists a subsequence that converges to each of those limits.
        Therefore, there exists two subsequences that converge to different limits.

    \end{proof}

    % Problem 2 Part D

    \begin{exern}{2D}

        If $x_n$ is monotone and contains a convergent subsequence, then $x_n$ converges.
    
    \end{exern}

    \begin{proof}
        
        False. I'm not justifying this because I just checked the homework assignment, and realized I only had to solve part A.
        But I'm still going to keep the rest of the answers here.

    \end{proof}

  \end{exern}

\end{document}
