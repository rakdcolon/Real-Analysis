\documentclass[boxes]{rutgers_hw}
\usepackage{rutgers}
% \usepackage[none]{hyphenat} % Use to avoid hyphens



\author{Rohan Karamel} % Enter your name
\netid{rak218} % Enter your NetID or comment out
% \collaborators{Leonhard Euler, Bernard Bolzano} % Enter your collaborators or comment out
\assignment{Homework 1} % Enter the assignment name
\date{\today} % Replace with due date
\course{Honors Real Analysis} % Enter the course name
\semester{Spring 2024} % Enter the semester
% \sectionnum{Honors} % Enter your section number
\instructor{Professor Sagun Chanillo} % Enter your professor's name
\institution{Rutgers University} % Enter your university

\begin{document}

\maketitle

% --------------------------------------------------------------

\begin{exern}{1.a} 
  Given $A, B \subseteq \mathbb{R}$, prove that if $\inf(A), \inf(B)$ finite, then $\inf(A+B) = \inf(A) + \inf(B)$.
\end{exern}
\begin{solution}
  Assuming the sets A and B nonempty, we know that \\ $\inf(A) + \inf(B) \le a + b \ \ \forall a \in A, b \in B $.
  Thus, $\inf(A) + \inf(B) \le \inf(A+B)$.

  \hfill \\
  Now, for all $\epsilon > 0$, there exists an $a \in A$ and $b \in B$ such that
  \[ a < \inf(A) + \frac{\epsilon}{2}\]
  \[ b < \inf(B) + \frac{\epsilon}{2}\]
  Adding these together yields
  \[ a + b < \inf(A) + \inf(B) + \epsilon\]
  We also know that,
  \[ \inf(A+B) \le a + b < \inf(A) + \inf(B) + \epsilon \]
  For all $\epsilon > 0$, we have
  \[ \inf(A + B) - \epsilon \le \inf(A) + \inf(B) \]
  We conclude,
  \[ \inf(A) + \inf(B) = \inf(A + B) \]
\end{solution}

% --------------------------------------------------------------
\pagebreak
% --------------------------------------------------------------

\begin{exern}{1.b.1} 
  Prove that (1.a) is true if $\inf(A)$ is infinite and $\inf(B)$ is finite.
\end{exern}
\begin{proof} 
  Assume for the sake of contradiction that $\inf(A+B)$ is finite. 
  By definition, we know that $\inf(A + B) \le a + b$ for all $a \in A$ and $b \in B$. \\
  We know that
  \[\inf(A+B) \le a + \inf(B) \ \ \forall a \in A\]
  \[\inf(A+B) - \inf(B) \le a \ \ \forall a \in A\]
  Recall that $\inf(A)$ is negative infinity, so there exists an $a \in A$ such that $a < M$ for any real number, M.
  Setting $M = \inf(A+B) - \inf(B)$, a contradiction occurs.
  Thus, $\inf(A+B)$ is infinite.
\end{proof}

% --------------------------------------------------------------

\begin{exern}{1.b.2} 
  Prove that (1.a) is true if $\inf(A)$ and $\inf(B)$ is infinite.
\end{exern}
\begin{proof}
  Assume for the sake of contradiction that $\inf(A+B)$ is finite. 
  By definition, we know that $\inf(A+B) \le a + b$ for all $a \in A$ and $b \in B$.
  Similarly, we have
  \[ \inf(A+B) - b \le a \ \ \ \forall a \in A, b \in B \]
  We know that because $\inf(A)$ is negative infinity, there exists an $a \in A$ such that $a < M$ for any real number, M.
  We can set $M = \inf(A+B) - b$.
  Now, we have that
  \[ M \le a \]
  And thus, a contradiction occurs.
  Therefore, $\inf(A+B)$ is infinite.
\end{proof}

% --------------------------------------------------------------
\pagebreak
% --------------------------------------------------------------

\begin{exern}{2}
  Let $a < b$ be real numbers and consider the set $T = \mathbb{Q}\cap[a,b]$. Show $\sup(T) = b$.
\end{exern}
\begin{proof}
  If $b$ is rational, then the proof is trivial. If $b$ is irrational, we proceed as follows.
  We know $b$ is an upper bound for $T$ because
  \[ \forall x \in \mathbb{Q}\cap[a,b], x \le b\]
  Now, to prove that $b = \sup(T)$, we need only show that
  \[ \forall \epsilon > 0, \exists r \in T:b - \epsilon < r\]
  From the density of the rationals in the reals we know that there exists a rational number, $r_0$, in the interval $[a,b]$.
  We can set $a = b - \epsilon$ for all epsilon positive to get
  \[ b - \epsilon < r_0 < b \]
  Thus, we have shown that $b$ is the least upper bound.
\end{proof}
\end{document}
