\documentclass[12pt,reqno]{amsart}

\usepackage{graphicx}

\usepackage{amssymb}
\usepackage{amsthm}
\theoremstyle{plain}

\newtheorem*{definition}{Definition}
\newtheorem*{axiom}{Axiom}
\newtheorem*{theorem}{Theorem}
\newtheorem*{corollary}{Corollary}
\newtheorem*{lemma}{Lemma}
\newtheorem*{example}{Example}
\newtheorem*{proposition}{Proposition}
\usepackage{lineno}

\title{Honors Real Analysis Lecture 2}
\author{Rohan Karamel}

\begin{document}

    \begin{abstract}
        This lecture covers sections 1.3 and 1.4 from the text, \textit{Understanding Analysis}.
        It delves into additional supremum/infimum theorems, topologies of open sets, denseness of the rationals in the reals, and existence proofs. 
    \end{abstract}

    \maketitle

    \begin{definition}
        Let $A \in \mathbb{R}$.
        We say that A is a bounded set if and only if there exists a positive, real number, M such that for any $a \in A, |a| \le M$ 
    \end{definition}

    Remarks:
    \begin{enumerate}
        \item $A \subseteq [-M, M]$ 
        \item $a \le M \implies$ A bounded above. 
        \item $a \ge -M \implies$ A bounded below.
        \item A set is bounded if and only if it is bounded above and below. 
    \end{enumerate}

    \begin{lemma}[1.3.8]
        Let $x_0 = \sup{A}$ and $x_0$ finite. One can find $a \in A$ such that 
        \[ x_0 - \epsilon < a < x_0 \]
    \end{lemma}

    Remarks: 
    \begin{enumerate}
        \item $\infty$ is just a symbol and is used as notation. The interval ${(a,\infty)} = \{ x \vert x > a\}$.
        \item The supremum can exist and be infinite.
    \end{enumerate}


    \begin{definition}
        Given $A \subseteq \mathbb{R}$, we say $\sup{A} = \infty$ if and only if for any $M > 0, \sup{A} > M$.
    \end{definition}

    \pagebreak

    \begin{theorem}[Consequence of Dirichlet's Theorem]
        Given an interval $(a,b)$ there exists a rational number, $r$, such that $r \in (a,b)$.
    \end{theorem}
    \begin{proof}
        Let $n \in \mathbb{N}$ such that, by the Archimedean Property, $\frac1n < b-a$
        Consider na, we can find $m \in \mathbb{N}$ such that $0 \le na < m$
        Let $m_0$ be the smallest natural number that satisfies that expression.
        Thus, $na < m_0$ and $m_0 - 1 \le na$. 
        We now have the following equivalent inequalities
        \[ a < \frac{m_0}{n}, \hspace{5 mm} \frac{m_0}{n} - \frac{1}{n} \le a\]
        Therefore, substituting b, 
        \[\frac{m_0}{n} \le a + \frac1n < a + (b-a) = b\]
        Finally, we have
        \[ a < \frac{m_0}{n} < b\]
        Because $m_0, n \in \mathbb{N}$, we have found a rational number between any interval $(a,b)$.
    \end{proof}

    \begin{definition}
        Given two sets $A,B \subseteq \mathbb{R}$. We define
        \[ A+B = \{ a + b \vert a \in A, b \in B \}\] 
    \end{definition}

    \begin{theorem}
        \[ \sup{(A+B)} = \sup{(A)} + \sup{(B)} \]
    \end{theorem}
    \begin{proof}
        $(\ \le\ )$ We begin by using the definition of supremum for the right hand side.
        \[ \forall a \in A,b \in B \hspace{4 mm} a + b \le \sup{(A)} + \sup{(B)}\]
        Therefore, $\sup(A+B) \le \sup(A) + \sup(B)$

        $(\ \ge\ )$ Suppose for all epsilon positive, there exists, a, an element of A, and, b, an element of B such that
        \[ \sup{(A)} - \frac{\epsilon}2 < a\]
        \[ \sup{(B)} - \frac{\epsilon}2 < b\]
        If we sum these two we get 
        \[ \sup{(A)} + \sup{(B)} - \epsilon < a + b \le \sup{(A+B)}\]
        Therefore, we have
        \[ \sup{(A)} + \sup{(B)} \le \sup{(A+B)} + \epsilon \]
        Thus, we conclude
        \[ \sup{(A)} + \sup{(B)} = \sup{(A+B)}\]
    \end{proof}

    \begin{theorem}[The Nested Interval Problem]
        Let $I_n = [a_n, b_n]$ such that ${\{I_n\}}_{n=1}^\infty$ is nested. Then $\cap^\infty_{n=1}I_n \neq \varnothing$.
    \end{theorem}
    \begin{proof}
        Consider the set $L = {\{ a_1 \le a_2 \le \dots \le a_n \le \dots \}}$. Where $a_i$ is the left-endpoint of $I_i$.
        Similarly, $R = {\{ \dots \ge b_n \ge \dots \ge b_2 \ge b_1\}}$. Where $b_i$ is the right-endpoint of $I_i$
        By the Axiom of Completeness, $\sup(a_n) = x_0$ exists.
        We know that $x_0 < b_n$, and we also know $\forall n,\ x_0 \in I_n$.
        So because $b_n$ is an upper bound and $x_0$ is the supremum, then $x_0 \le b_n$. Therefore $\cap^\infty_{n=1}I_n \neq \varnothing$.
    \end{proof}

    \begin{definition}[Topology]
        Given a set, X. We say $\mathcal{F}$ is a topology on X If
        \begin{enumerate}
            \item $A \in \mathcal{F}, A \subseteq X$
            \item $X, \varnothing \in \mathcal{F}$
            \item ${\{A_\alpha\}}_{\alpha\in I} \subseteq \mathcal{F} \implies \cup_{\alpha \in I}A_\alpha \in \mathcal{F}$
            \item $A_i \in \mathcal{F} \implies \cap^n_{i=1}A_i \in \mathcal{F}$
        \end{enumerate}
    \end{definition}

    \begin{definition}
        A set $B$ is closed if and only if $X\setminus B$ is open.
    \end{definition}

    \begin{definition}
        A set $S \subseteq \mathbb R$, we say S is dense in the reals if and only if for any open interval, one can find $s \in (a,b)$.
    \end{definition}

    Remark: Therefore, $\mathbb{Q}$ is dense in $\mathbb{R}$.

    \begin{theorem}
        $\sqrt2$ exists.
    \end{theorem}
    \begin{proof}
        Consider the set $S = {\{  x \vert x^2 M 2 \}}$.
        We know S is nonempty since $1 \in S$. 
        Next, we know S is bounded because if $x_0 > 4$, then $x_0^2 > 16$.
        Now, for any epsilon positive, there exists an element, y, in S such that $x_i - \epsilon < y$.
        So,
        \[ {(x_0 - \epsilon)}^2 < y^2 < 2 \implies x_0^2 + \epsilon^2 - 2\epsilon x < 2 \implies x_0^2 < 2\]
        Therefore, the supremum of this set is the square root of 2. Therefore, it must exist.
    \end{proof}

\end{document}