\documentclass[12pt,reqno]{amsart}

\usepackage{graphicx}

\usepackage{amssymb}
\usepackage{amsthm}
\theoremstyle{plain}

%% this allows for theorems which are not automatically numbered

\newtheorem*{definition}{Definition}
\newtheorem*{axiom}{Axiom}
\newtheorem*{theorem}{Theorem}
\newtheorem*{corollary}{Corollary}
\newtheorem*{lemma}{Lemma}
\newtheorem*{example}{Example}
\newtheorem*{proposition}{Proposition}
\usepackage{lineno}

\title{Honors Real Analysis Lecture 1}
\author{Rohan Karamel}

\begin{document}

    \begin{abstract}
        This lecture covers a review of Chapter 1 and 2 to prepare for the upcoming first midterm.
    \end{abstract}

    \maketitle

    \begin{proposition}
        Except for finitely many, $a_n \le a + \epsilon, \forall n \ge N$.
    \end{proposition}

    \begin{proof}
        Assume there are infinitely many $a_n$, such that $a_n > a + \epsilon$.
        So we have $a_{n_1}, a_{n_2}, \dots, a_{n_k}, \dots$ such that (1): $a_{n_k} > a + \epsilon$
        We proceed by cases:
        \textbf{Case 1}: $a_{n_k} \le M$ \\
        $a_{n_k} > a + \epsilon$ is given
        By Bolzano-Weierstrass, there exists
        \[ \lim_{j \to \infty} a_{n_k} = a_0\]
        Using (1) and order theorem,
        \[ \lim_{j \to \infty} a_{n_k} = a_0 \ge a + \epsilon\]
        Now $a_0 \in S$
        \[ a = \sup S \ge a_0 \ge a + \epsilon\]
        Thus, a contradiction forms as epsilon must be positive.
    \end{proof}

    \begin{proposition}
        \[ \limsup_{n\to\infty} (a_n + b_n) \le \limsup_{n\to\infty} a_n + \limsup_{n\to\infty} b_n \]
        This is an equality if both sequences converge. For example:
        \[ a_n = {(-1)}^n, b_n = {(-1)}^{n+1} \]
        \[ a_n + b_n = 0 + 0 + \dots \]
        \[ \limsup_{n\to\infty} (a_n + b_n) = 0 \]
        \[ 0 \le 1 + 1 = 2\]
    \end{proposition}

    \begin{proof}
        We proceed by cases. \\
        \textbf{Case 1}: Assume $a = \limsup_{n\to\infty}a_n, b = \limsup_{n\to\infty}b_n$ and $a,b$ both finite. \\
            We recall, for every $\epsilon > 0$
            \[a_n \le a + \epsilon, \forall n \ge N_1 \]
            \[b_n \le b + \epsilon, \forall n \ge N_2 \]
            So for $n \ge \max{N_1, N_2} = N$
            \[ a_n + b_n \le a + b + 2\epsilon, \forall n \ge N \]
            So for any subsequence
            \[ a_{n_k} + b_{n_k} \le a + b + 2\epsilon, \forall k \ge k_0 \]
            Thus,
            \[ \lim_{k \to \infty} (a_{n_k} + b_{n_k} \le a + b + 2\epsilon)\]
            via order theorem, we have
            \[\sup S \le a + b + 2\epsilon \]
            \[\lim_{n \to \infty} (a_n + b_n) \le a + b + 2\epsilon, \forall \epsilon > 0 \]

        \textbf{Case 2}: Given any $ M > 0 $,
            \[ a_{n_k} > M, \forall k \ge k_0 \iff \limsup_{n\to\infty} a_n = +\infty\]
            Assume $b_n \le M_1$. Then, note that
            \[ a_{n_k} + b_{n_k} \ge M - M_1 \ge \frac M 2 \]
            If M is very large, $M \ge 2M_1$. Thus,
            \[ a_{n_k} + b_{n_k} > \frac M 2, \forall k \ge k_0 \]
            And thus,
            \[ \limsup (a_n + b_n) = \infty \]
        
        \textbf{Case 3}: Given any $ M > 0 $,
            \[ a_{n_k},b_{n_k} > M, \forall k \ge k_0 \iff \limsup_{n\to\infty} a_n = +\infty, \limsup_{n\to\infty} b_n = +\infty\]
            The proof is left as an exercise to the reader.

    \end{proof}

    \begin{theorem}
        Let $S \subseteq \mathbb{N}$. Then S is countable
    \end{theorem}

    \begin{proof}
        Use the well-ordering theorem. There exists $x_{i_1} \in S$, a least element. Note S has no repetition.
        Now consider $S \setminus {x_{i_1}}$.
        Then, there exists a least element, $x_{i_2} \in S \setminus {x_{i_1}}$.
        And continue to form $\{{x_{i_1}}, {x_{i_2}}, \dots, {x_{i_k}}, \dots\}$.
        Construct $f:\mathbb{N} \to S$. 
        $f$ is injective as we threw repetitions out, therefore each input has a unique output.
        $f$ is surjective as we can achieve every natural number by looking at the index  of $i$. 
        Likewise, we can get every index of $i$ with every natural number.
        Thus, we have a bijective function that maps the naturals to S. Therefore, S is countable.

    \end{proof}

    \begin{proposition}
        This series diverges:
        \[ \sum_{n=1}^\infty \frac 1 {\sqrt{6n^2 + 10n} + \sqrt{n^2 + 3}}\] 
    \end{proposition}

    \begin{proof}
        \[ \frac 1 {\sqrt{6n^2 + 10n} + \sqrt{n^2 + 3}} \ge \frac 1 {\sqrt{6n^2 + 10n^2} + \sqrt{n^2 + 3n^2}} \]

    \end{proof}

    Tests we can use to determine convergence and divergence for series:
    \begin{enumerate}
        \item Comparison Test
        \item Cauchy Condensation Theorem ($\sum r^k a_{r^k}$)
        \item Ratio Test (Utilize comparison test with a geometric series)
        \item Root Test (Utilize comparison test with a geometric series)
        \item Partial Summation $\to$ Dirichlet's Test $\to$ Alternating Series Test.
    \end{enumerate}

    How to solve for $\sum \sin(kx)$: \\
    \[ \sum_{k=0}^n e^{ikx} = \left[ \sum \cos(kx) \right] + \left[ i\sum \sin(kx) \right]\]
    \[ \frac {1 - e^{i(n+1)x}}{1-e^{ix}} \]

\end{document}