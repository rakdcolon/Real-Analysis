\documentclass[boxes,sansserif]{rutgers_hw}
\usepackage{rutgers}
% \usepackage[none]{hyphenat} % Use to avoid hyphens



\author{Rohan Karamel} % Enter your name
\netid{rak218} % Enter your NetID or comment out
% \collaborators{Leonhard Euler, Bernard Bolzano} % Enter your collaborators or comment out
\assignment{Workshop 1} % Enter the assignment name
\date{\today} % Replace with due date
\course{ Real Analysis} % Enter the course name
\semester{Spring 2024} % Enter the semester
\sectionnum{H} % Enter your section number
\instructor{Professor Sagun Chanillo} % Enter your professor's name
\institution{Rutgers University} % Enter your university

\begin{document}

\maketitle

% --------------------------------------------------------------

\begin{exern}{1.a} 
  Given $a_1, a_2, \dots, a_n > 0$ all positive numbers. \\ Prove that $a^2_1 + a^2_2 \ge 2a_1a_2$
\end{exern}

\begin{proof} 
  We begin by subtracting $2a_1a_2$ from both sides of the inequality to get $a^2_1 + a^2_2 - 2a_1a_2 \ge 0$. 
  We can factor this to get ${(a_1 - a_2)}^2 \ge 0$. 
  Since $a_1$ and $a_2$ are both positive, we know that ${(a_1 - a_2)}^2$ is also positive. 
  Thus, we have proven that $a^2_1 + a^2_2 \ge 2a_1a_2$.
\end{proof}

% --------------------------------------------------------------

\begin{exern}{1.b.1} 
    Prove that $\frac{a_1}{a_2} + \frac{a_2}{a_1} \ge 2$
\end{exern}

\begin{proof} 
    We begin by multiplying both sides of the inequality by $a_1a_2$ to get $a^2_1 + a^2_2 \ge 2a_1a_2$. 
    This is identical to the inequality we proved in part (a), so we know that it is true.
\end{proof}

% --------------------------------------------------------------

\begin{exern}{1.b.2}
    Prove that ${(a_1 + a_2 + a_3)}{(\frac{1}{a_1} + \frac{1}{a_2} + \frac{1}{a_3})} \ge 9$
\end{exern}

\begin{proof}
    We begin by expanding the left hand side.
    \[ 3 + \frac{a_1}{a_2} + \frac{a_1}{a_3} + \frac{a_2}{a_1} + \frac{a_2}{a_3} + \frac{a_3}{a_1} + \frac{a_3}{a_2} \ge 9\]
    We can simplify and group terms to get
    \[ {(\frac{a_1}{a_2} + \frac{a_2}{a_1})} + {(\frac{a_1}{a_3} + \frac{a_3}{a_1})} + {(\frac{a_2}{a_3} + \frac{a_3}{a_2})} \ge 6\]
    We can use the inequality we proved in part (1.b.1) to show that each of the terms on the left hand side is greater than or equal to 2. 
    Thus, each of the terms on the left hand side is greater than or equal to 2, and the sum of the terms is greater than or equal to 6.
    Therefore, we have proven that ${(a_1 + a_2 + a_3)}{(\frac{1}{a_1} + \frac{1}{a_2} + \frac{1}{a_3})} \ge 9$.
\end{proof}

% --------------------------------------------------------------

\begin{exern}{1.c} 
    Prove that ${(\Sigma^n_{i=1}{a_i})}{(\Sigma^n_{j=1}{\frac{1}{a_j}})} \ge n^2$
\end{exern}

\begin{proof} 
    We can rewrite the two sums as $\Sigma^n_{i=1}{\Sigma^n_{j=1}{\frac{a_i}{a_j}}}$. 
    We can group two terms in the sum by rewriting it as follows:
    \[ n + \Sigma^n_{i=1}\Sigma^n_{j=i+1}{\left(\frac{a_i}{a_j} + \frac{a_j}{a_i}\right)}\]
    The term of $n$ appears for the excluded terms of the sum where $i = j$.
    We can use the inequality we proved in part (1.b.1) to show that each of the terms in the sum is greater than or equal to 2.
    Therefore,
    \[ n + \Sigma^n_{i=1}\Sigma^n_{j=i+1}{\left(\frac{a_i}{a_j} + \frac{a_j}{a_i}\right)} \ge n + \Sigma^n_{i=1}\left(\Sigma^n_{j=i+1}{2}\right)\]
    We can count the number of terms in the sum on the right hand side by counting the number of pairs of $i$ and $j$ such that $i < j$.
    This is, in fact, $n \choose 2$, because, we need to choose two distinct $a_k$'s. Because each term is 2, we multiply this by 2 to get $n(n-1)$. 
    Therefore,
    \[ n + \Sigma^n_{i=1}\Sigma^n_{j=i+1}{\left(\frac{a_i}{a_j} + \frac{a_j}{a_i}\right)} \ge n + n(n-1) = n^2\]
    And we are done.
\end{proof}

% --------------------------------------------------------------

\begin{exern}{3.c}
  Prove for any $k \in \mathbb{N}$, there exists $N_3$ such that $k^n \ge {n!}$.
\end{exern}
\begin{proof} 
  Let $N_3 = 3k$. We proceed by induction on $n$. 
  Let \[S = \{n \mid k^n \le n! \lor n < N_3\}\] 
  Our goal is to prove $S = \mathbb{N}$.
  \begin{induction}
    \begin{basecase}
      We first need to show $N_3 \in S$. We can rewrite the inequality as $k^{3k} \le {(3k)!}$. On the LHS, we have 3k factors of k. On the RHS, we have $3k$ factors, $2k+1$ of which is greater than or equal to k. Because there are double as many terms greater than k than less than k, we can reasonably assume that this statement is true. 
    \end{basecase}
    \begin{indhyp}
      Now, we assume that $n \in S: n \ge N_3$ and will show that $n+1 \in S$.
    \end{indhyp}
    \begin{indstep}
      From our assumption, we know
      \[ k^n \le n! \]
      Multiplying both sides by $k$ gives us
      \[ k^{n+1} \le k(n!) \]
      We also know that, because $n > 3k$, then
      \[ k(n!) \le (n+1)! \]
      Thus, we have
      \[ k^{n+1} \le (n+1)! \]
      Therefore, $n+1 \in S$.
    \end{indstep}
    Thus by the principle of mathematical induction, we have proven our theorem.
  \end{induction}
\end{proof}

\begin{description}
    \item Note: Because we have proven this statement for all k, Problem 3.a and 3.b are also proven.
\end{description}




\end{document}
