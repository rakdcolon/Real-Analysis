\documentclass[12pt,reqno]{amsart}

\usepackage{graphicx}

\usepackage{amssymb}
\usepackage{amsthm}
\theoremstyle{plain}

\newtheorem*{theorem*}{Theorem}
\newtheorem{proposition*}{Proposition}
%% this allows for theorems which are not automatically numbered

\newtheorem{definition}{Definition}
\newtheorem{axiom}{Axiom}
\newtheorem{theorem}{Theorem}
\newtheorem{corollary}{Corollary}
\newtheorem{lemma}{Lemma}
\newtheorem{example}{Example}
\newtheorem{proposition}{Proposition}
\usepackage{lineno}

%% The above lines are for formatting.  In general, you will not want to change these.


\title{Honors Real Analysis Lecture 1}

\author{Rohan Karamel}

\begin{document}

\begin{abstract}
    We begin with a proof of the extended triangle inequality by induction. Then, move on to supremums an infimums. We also cover the Axiom of Completeness and the Archimedean Property. Lastly, we introduce the denseness of the rational numbers.
\end{abstract}

\maketitle

\begin{proposition}[Extended Triangle Inequality]
    \[ |\sum_{k=1}^n\pm a_k| \le \sum_{k=1}^n |a_k| \]
\end{proposition}


\begin{proof}
    We proceed by induction.
    Our base step is n = 2, which is true from the triangle inequality.

    \[ |a + b| \le |a| + |b| \]

    Let $P(n)$ be the statement, $|\sum_{k=1}^n\pm a_k| \le \sum_{k=1}^n |a_k|$. 
    We will prove that for any natural number, $n$, $P(n) \implies P(n+1)$. 
    Let us assume $P(n)$ is true and show $P(n+1)$ follows. \\
    Letting $x = \pm a_{n+1}$ and $y = \sum_{k=1}^{n+1}\pm a_k$, we have

    \[ |\sum_{k=1}^{n+1}\pm a_k| = |x + y| \le |x| + |y| = |\pm a_{n+1}| + |\sum_{k=1}^{n}\pm a_k| \]

    By our induction hypothesis, we know

    \[ |\pm a_{n+1}| + |\sum_{k=1}^{n}\pm a_k| = |a_{n+1}| + \sum_{k=1}^n|a_k| = \sum_{k=1}^{n+1}|a_k| \]

    Thus, we have $P(n+1)$

    \[ |\sum_{k=1}^{n+1}\pm a_k| \le \sum_{k=1}^{n+1}|a_k|\]

\end{proof}

\pagebreak

\begin{proposition}
    \[ ||a| - |b|| \le |a+b| \]
\end{proposition}

\begin{proof}
    By definition, 
    \[ ||a| - |b|| = 
    \begin{cases}
        |a| - |b|, |a| \ge |b| \\
        |b| - |a|, |b| \ge |a|
    \end{cases}\]
    We need only check that $|a| - |b| \le |a+b|$ and $|b| - |a| \le |a+b|$.
    Start with
    \[ |a| = |a + b - b| \]
    \[ |a + b - b| \le |a + b| + |-b| = |a + b| + |b| \]
    Thus, we have
    \[ |a| \le |a + b| + |b| \implies |a| - |b| \le |a + b| \]
    Likewise,
    \[ |b| = |b+a-a| \le |b+a| + |a| \implies |b|-|a| \le |a+b|\]
    Because we have covered all cases, it follows that
    \[ ||a| - |b|| \le |a+b| \]
\end{proof}

\begin{corollary}
    \[ ||a| - |b|| \le |a-b| \]
\end{corollary}
\begin{proof}
    From the previous proposition, we have
    \[ ||a| - |b|| \le |a+b| \]
    Let $b = -b$
    \[ ||a| - |-b|| \le |a+(-b)|\]
    Therefore,
    \[ ||a| - |b|| \le |a-b|\]
\end{proof}

\pagebreak

\begin{definition}
    Let A be a non-empty subset of the real numbers. The supremum of A is the least upper bound of the set A.
\end{definition}

Remark: $\sup{A}$/$\inf{A}$ need not be an element of A. Take, for example, $A = (0,1)$: $\sup{A} = 1, \sup{A} \not \in A$.

\begin{proposition}
    Let $A \subseteq \mathbb{R}$. Let $x_0$ be an upper bound for A. 
    Then $x_0 = \sup{A} \iff $ for any $ \epsilon > 0$, there exists an element $a \in A$ such that for $x_0 - \epsilon \le a$.
\end{proposition}

\begin{proof} 
    $(\implies)$ 
    Let $x_0 = \sup{A}$. For the sake of contradiction, assume that there is an $\epsilon_0$ such that for all $a \in A$, $x_0 - \epsilon \le a$.
    Therefore, by definition, $x_0 - \epsilon_0$ is an upper bound. 
    However, $x_0$ is strictly less than $x_0 - \epsilon$.
    Thus, a contradiction is formed because we assumed $x_0$ was the least upper bound.

    $(\ \Longleftarrow \ )$ We know for any $a \in A, a \le x_0$ since $x_0$ is an upper bound.
    Next, let us pick a number smaller than $x_0$, say $x_0 - \epsilon$. 
    Our hypothesis says, there exists $a \in A$ such that $x_0 - \epsilon < a$.
    Therefore, it cannot be an upper bound. Because $x_0$ is an upper bound and any number smaller than $x_0$ is not an upper bound, $x_0$ is $\sup{A}$.
\end{proof}

\vspace{10 mm}

\begin{proposition}
    \[ \alpha = \sup{(-A)} = -\inf{(A)} = \beta \]
\end{proposition}
\begin{proof}
    Our strategy is to first show (1) $\alpha \le \beta$, then show that \\(2) $\alpha \ge \beta$. Therefore, $\alpha = \beta$. \\
    (1): For any $\epsilon > 0$, one finds $x_0 \in A$ such that
    \[ \alpha - \epsilon < -x_0\]
    \[ \epsilon - \alpha > x_0 \]
    But $\inf{A} \le x_0 \le \epsilon - \alpha$. Therefore,
    \[ -\alpha + \epsilon > \inf{(A)}\]
    \[ \epsilon - \alpha < -\inf{(A)} = \beta\]
    So, $\alpha - \beta < \epsilon$. \\
    (2): By definition,
    \[ \forall x \in A, -x \le \alpha \implies x \ge \alpha\]
    So, $-\alpha$ is a lower bound of A. Therefore, $\inf{(A)} \ge -\alpha$
    Equivalently, $\alpha \ge \beta$. \\
    Therefore, by (1) and (2), $\alpha = \beta$.
\end{proof}

\pagebreak

\begin{axiom}[Axiom of Completeness]
    Given a non-empty subset of $\mathbb{R}$. If that set is bounded above, then its supremum exists.
\end{axiom}

\begin{theorem}[Archimedean Property] \hfill \\
    (1) Let $x \in \mathbb{R}$, then there exists a natural number, $n$, such that $x < n$ \\
    (2) Let $x \in \mathbb{R}$, then there exists an integer such that $n < x < n+1$ \\
    (3) Let $x > 0$, then there exists a natural number, $n$, such that $\frac1n < x$
\end{theorem}

\begin{proof} \hfill
    
    (1): Assume, for the sake of contradiction, that for any natural number, n, $n \le x$.
    Define the set S as follows,
    \[ S = \{ n \vert n \le x \} \]
    S is bounded above, so by completeness, the supremum exists.
    Let $x_0 = \sup S$
    So, there exists an $n_0 \in \mathbb{N}$, such that 
    \[ x_0 - 1 < n_0 \]
    This also shows that 
    \[ x_0 < n_0 + 1 \in \mathbb{N} \]
    Thus a contradiction forms as there exists a natural number greater than $x_0$.
    
    (2): Given any $x$ and, without loss of generality, let $x > 0$. \\
    From (1), we know there exists a natural number, $n$, such that $x < n$.
    Let $S = \{ n \vert n \le x \}$. 
    By the well-ordering principle, we know that $n_0$ is the smallest element of S. 
    So,
    \[ n_0 - 1 < x < n_0 \]

    (3): Consider $\frac1x = z \in \mathbb{R}$
    From (1) we have $n \in \mathbb{N}$ so that 
    \[ \frac1x = z < n \implies \frac1n < x \] 
\end{proof}

\begin{theorem}[Consequence of Dirichlet's Theorem]
    Let ${(a,b)}$ be any interval, then there exists a rational number, $r$, such that $r \in {(a,b)}$.\  i.e. $a < r < b$.
\end{theorem}
\begin{proof}
    The proof is left as an exercise to the reader.
\end{proof}

\end{document}